\documentclass{article}
\usepackage[utf8]{inputenc}

\title{Movie Metadata Exploratory Data Analysis}
\author{Tianyuan Zheng}
\date{October 2, 2020}

\begin{document}
\maketitle

\section{Introduction}
The data of this exploratory data analysis (EDA) contains information including actors, movie duration, genres m, IMDB ratings and so on. The general purpose for this EDA is to explore the features of most popular movies. The popularity is evaluated by IMDB scores, number of movies, and gross amount collected during theatrical run for different directors, actors, and genres.

\section{Figures}

\begin{figure}[H]
\includegraphics[width=\textwidth]{f1.png}
\end{figure}

This graph examines the directors with most films in this data-set. The number of movies is used as an indicator of popularity as popular directors tend to film more with enough market support. However, the graph also indicates a small variance between the number of movies and the average IMDB rates for most popular directors.

\begin{figure}[H]
\includegraphics[width=\textwidth]{f2.png}
\end{figure}

\begin{figure}[H]
\includegraphics[width=\textwidth]{f2.png}
\end{figure}
\begin{figure}[H]
\includegraphics[width=\textwidth]{f2.png}
\end{figure}
\begin{figure}[H]
\includegraphics[width=\textwidth]{f2.png}
\end{figure}

\section{Data Source}
https://www.kaggle.com/roshansharma/movies-meta-data
\end{document}
